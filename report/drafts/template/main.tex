%% %%%%%%%%%%%%%%%%%%%%%%%%%%%%%%%%%%%%%%%%%%%%%%%%%
%% Template for a conference paper, prepared for the
%% Food and Resource Economics Department - IFAS
%% UNIVERSITY OF FLORIDA
%% %%%%%%%%%%%%%%%%%%%%%%%%%%%%%%%%%%%%%%%%%%%%%%%%%
%% Version 1.0 // November 2019
%% %%%%%%%%%%%%%%%%%%%%%%%%%%%%%%%%%%%%%%%%%%%%%%%%%
%% Ariel Soto-Caro
%%  - asotocaro@ufl.edu
%%  - arielsotocaro@gmail.com
%% %%%%%%%%%%%%%%%%%%%%%%%%%%%%%%%%%%%%%%%%%%%%%%%%%
\documentclass[11pt]{article}
\usepackage{UF_FRED_paper_style}
\usepackage{imakeidx}
\makeindex
\usepackage{lipsum}  %% Package to create dummy text (comment or erase before start)

%% ===============================================
%% Setting the line spacing (3 options: only pick one)
% \doublespacing
% \singlespacing
\onehalfspacing
%% ===============================================

\setlength{\droptitle}{-5em} %% Don't touch

\title{Background}
\author{Giedrius Mirklys}
\date{\today}
\begin{document}

\maketitle

% --------------------
\section{Background}
\label{sec:background}
% --------------------

Parkinson's Disease has been a subject of extensive research due to its complex neurophysiological underpinnings and
the significant impact it has on patients' quality of life. It originates differently in each patient, making it
a hurdle for an effective treatment. Recent advancements in data-driven therapeutic neurostimulation have shown promise in addressing the
symptoms of Parkinson's Disease by leveraging patient-specific data to tailor interventions. However, creating effective
neurostimulation protocols requires a deep understanding of neurodynamics that govern behavior in Parkinson's patients.

This section delves into the neuroelectrophysiological aspects of PD that could be learned by the neurodynamical models to
drive therapeutic neurostimulation. It also provides a theoretical background of PD pathophysiology and discusses
treatment approaches that utilize data-driven neurostimulation techniques. Finally, it explores the role of neurodynamical models
in understanding and predicting behavioral outcomes in Parkinson's Disease while outlining two frameworks that could be used
to model neurodynamics in PD.

\subsection{Neurophysiology of Parkinson's Disease}
\label{subsec:neurophysiology-of-parkinson's-disease}

Primarily, reduced availability of dopamine in the basal ganglia (BG) is the main precursor to the motor symptoms of PD.
Dopamine depletion leads to an imbalance in major pathways of the BG, leading to impairment in initiating
and controlling movements. Connectivity between the BG, thalamus, and motor cortex becomes disrupted which altogether
orchestrates the motor functions. Since there are multiple pathways and relays passing through these regions, planned actions
need to be filtered (the indirect pathway: driven by the connection to thalamus) and refined (the direct pathway: running
through BG and motor cortex) to produce smooth and coordinated movements, and any other neural activity is suppressed (the hyperdirect
pathway: connecting the subthalamic nucleus to the globus pallidus).

It is largely unknown why dopamine-producing neurons in the substantia pars compacta (SNc) degenerate in PD, but it is
believed that a combination of genetic and environmental factors, such as pollution or bad nutrition, contribute to this process.
And its onset is usually marked by the presence of Lewy bodies, which are abnormal aggregates of the protein alpha-synuclein
within neurons. These aggregates are thought to disrupt normal cellular functions and contribute to neuronal death. As the disease progresses,
the loss of dopaminergic neurons leads to drastic neurophysiological changes, including altered neurotransmitter levels,
synaptic dysfunction, and changes in neural circuitry. Thus, the Cortico-Basal Ganglia-Thalamo-Cortical (CBGTC) loop, encompassing
the pathways mentioned earlier, becomes impaired, and individuals start experiencing tremors, rigidity, slowness of movement, or even cognitive decline.
Rigidity and bradykinesia are linked to increased inhibitory output from the BG to the thalamus, which reduces excitatory input to the motor cortex.
Cognitive symptoms are associated with dysfunctions in the prefrontal cortex and its connections with the BG and use of dopamine in these areas.
Tremors are believed to arise from abnormal oscillatory activity in the BG-thalamocortical circuits.

Oscillatory activity in the beta frequency band (13-30 Hz) is theorized to characterize the proper functioning of the CBGTC loop.
In PD, beta osccilations become excessively synchronized, prolonged, and amplified, while healthy patients exhibit transient beta bursts.
However, this is mostly seen in the averaged activity acrross multiple patients and may not be present in all individuals with PD.
Some studies suggest gamma oscillations (30-100 Hz) may also play a role in PD pathophysiology, particularly in relation to motor symptoms. Or,
increased beta power. Therefore, solutions targeting improper functioning of the CBGTC loop should be personalized to each patient's
neurophysiological profile.


\subsection{Data-driven Therapeutic Neurostimulation for Parkinson's Disease}
\label{subsec:data-driven-therapeutic-neurostimulation-for-parkinson's-disease}

\subsection{Neurodynamical Models of Behavior}
\label{subsec:neurodynamical-models-of-bahavior}

\medskip

\bibliography{references.bib} 

\newpage
\end{document}