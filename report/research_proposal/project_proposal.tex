\documentclass[12pt, a4paper]{article}

\usepackage[margin=1in, headheight=14.5pt]{geometry}
\usepackage[utf8]{inputenc}
\usepackage[T1]{fontenc}
\usepackage{times}
\usepackage{ragged2e} 
\usepackage{csquotes} 
\usepackage{amssymb}
\usepackage{amsmath}
\usepackage{booktabs} 
\usepackage{tabularx}
\usepackage{graphicx}
\usepackage{parskip}
\usepackage[style=apa, backend=biber]{biblatex}

\addbibresource{proposal_bibliography.bib} 
\addbibresource{proposal_manual_bibliography.bib}

\usepackage{titlesec}

\titleformat{\section}{\normalfont\fontsize{12}{15}\bfseries}{\thesection. }{0em}{}
\titleformat{\subsection}{\normalfont\fontsize{12}{15}\itshape}{\thesubsection. }{0em}{}
\titleformat{\subsubsection}{\normalfont\fontsize{12}{15}\mdseries}{\thesubsubsection. }{0em}{}
\titlespacing*{\section}{0pt}{3.5ex plus 1ex minus .2ex}{2.3ex plus .2ex}
\titlespacing*{\subsection}{0pt}{3ex plus 1ex minus .2ex}{1.5ex plus .2ex}
\titlespacing*{\subsubsection}{0pt}{2ex plus 0.5ex minus .2ex}{1ex plus .2ex}

\usepackage{hyperref}
\hypersetup{
    colorlinks=true,
    linkcolor=black,
    citecolor=blue,
    urlcolor=blue,
    pdftitle={Extended Research Project Proposal},
    pdfauthor={Giedrius Mirklys},
}


\begin{document}

\title{\textbf{PROJECT DESCRIPTION} \\ \vspace{1.5em} \large Modeling Latent Neural Dynamics to Decode Brain States and Motor Behavior in Parkinson’s Disease}
\author{Giedrius Mirklys \\
    \normalsize{Student Number: s1101773}}
\date{\today} % Or a specific date

\maketitle
\RaggedRight


\section{ABSTRACT}
Parkinson’s disease (PD) is a neurological disorder where motor symptoms are linked to pathological beta-band oscillations within cortico-basal ganglia circuits. While continuous Deep Brain Stimulation (DBS) is an effective treatment, its non-adaptive nature can lead to side effects, motivating a shift toward adaptive DBS (aDBS). The development of such therapies hinges on models that can interpret neural signals in real-time to guide stimulation. The linear Preferential Subspace Identification (PSID) and nonlinear Dissociative Prioritized Analysis of Dynamics (DPAD) frameworks have been successful in modeling non-pathological neural dynamics with animal data. However, their applicability for decoding complex, pathological signals in human neurological disorders remains unexplored. This thesis will investigate the utility of these models for decoding brain states and behavior in PD. The primary objectives are to characterize model performance across three tasks: $(1)$ cross-modal prediction of ECoG from LFP signals, $(2)$ classification of discrete DBS ON/OFF brain states, and $(3)$ continuous decoding of motor behavior. By applying these frameworks to simultaneous LFP and ECoG data, this master's thesis research project aims to develop a model capable of extracting behaviorally relevant dynamics from multiplexed neural signals characteristic of PD. The findings may offer insights into the neural dynamics of the disease, informing how biomarkers for closed-loop stimulation could be derived from latent model states.

\section{PROJECT DESCRIPTION}
% max 2000 words
% Use \parencite{key} for parenthetical citations, e.g., (Author, 2023).
% Use \textcite{key} for narrative citations, e.g., Author (2023) demonstrates.

According to the \textcite{who_parkinson_2023}, rates of Parkinson's disease (PD) increased twice over the last 25 years, with an estimated 8.5 million people affected worldwide in 2019. It manifests in slowness of movement, rigidity, and tremor, as well as cognitive decline, originating primarily from genetic and environmental factors \parencite{ben-shlomoEpidemiologyParkinsonsDisease2024}. In PD, the primary pathology is the loss of dopaminergic neurons in the substantia nigra, which substantially disrupts communication within the cortico-basal ganglia-thalamo-cortical loop.  Neurophysiologically, \textcite{tinkhauserBetaBurstDynamics2017} reported that the prolonged oscillations in the beta frequency band ($13-30$ Hz) are indicative of such a disruption, and its power has been correlated with the severity of the motor symptoms, including akinesia and rigidity, whereas in the healthy brain, the beta oscillations typically occur as transient short bursts during motor execution.


Therapeutic strategies aim to correct these disturbances at the network level, reducing the beta frequency synchrony within the structures in basal ganglia and between the motor cortex and the basal ganglia itself \parencite{tinkhauserBetaBurstDynamics2017, paulsCorticalBetaBurst2022}. Pharmacological treatments, primarily Levodopa which increases levels of dopamine, have been shown to suppress prolonged beta oscillations, improving motor performance. For patients with more advanced symptoms, Deep Brain Stimulation (DBS) offers an effective alternative by implanting electrodes in the subthalamic nucleus (STN) within the basal ganglia. It is theorized that DBS imposes a new, high-frequency electrical pattern that acts as an \textit{informational lesion}, overriding the pathological beta rhythm and preventing its propagation through the network \parencite{chikenMechanismDeepBrain2016,mcintyreNetworkPerspectivesMechanisms2010}. Additionally, as highlighted by \textcite{wuComputationalModelsAdvance2024}, the effects of DBS are multifaceted, inducing outcomes like synaptic plasticity and broader neural reorganization. However, the continuous, non-adaptive nature of the stimulation can lead to side effects, which are often attributed to the spread of electrical current to adjacent neural structures not targeted for therapy \parencite{zarzyckiStimulationinducedSideEffects2020}.

This limitation has motivated the development of adaptive DBS (aDBS), a closed-loop approach that aims to deliver stimulation only when needed. The effectiveness of aDBS hinges on the real-time identification of reliable biomarkers of symptom states from complex, high-dimensional neural data. While beta oscillations are a leading biomarker, it remains debated whether they are causal to motor symptoms or are an epiphenomenon, and whether patient-specific biomarkers involving other spectral features, such as beta phase-amplitude coupling, or even other frequency bands might be more effective \parencite{swannGammaOscillationsHyperkinetic2016, wuComputationalModelsAdvance2024}. The rationale for aDBS is that selectively targeting specific personalized neuromarkers, where the symptoms are pronounced the most, could enable a more efficient, closed-loop approach to stimulation, potentially reducing side effects \parencite{littleAdaptiveDeepBrain2013}. Computational models are therefore essential for learning the mapping between neural signals and clinical states, and for identifying robust neuromarkers to drive aDBS systems. Furthermore, since tuning aDBS controllers on patients is impractical, models provide a means to evaluate control strategies in silico.


\textcite{saniModelingBehaviorallyRelevant2021} described two standard approaches to modeling neural dynamics: Neural Dynamic Modeling (NDM) and Representational Modeling (RM). NDM typically learns a latent state that best predicts future neural activity from past neural activity, making it agnostic to behavior. Conversely, RM often models the dynamics of behavior itself, predicting future behavior from past behavior, and then relates this to neural activity, making it agnostic to the intrinsic dynamics of the neural signals. Neither approach is explicitly designed to isolate the dynamics that are shared between neural activity and behavior. This leads to the challenge of separating behaviorally relevant dynamics, which are the neural patterns that co-vary with and are predictive of behavior, from the vast background of behaviorally irrelevant dynamics related to other internal states and cognitive processes. Indeed, \textcite{wuMixedSelectivitySubthalamic2025} elaborated that neural populations in motor-related areas, especially STN, are known to multiplex signals for numerous variables simultaneously, making it critical to disentangle the specific dynamics related to the behavior of interest.

This thesis will systematically evaluate two advanced data-driven modeling frameworks for decoding latent neural dynamics in Parkinson’s disease (PD): the linear Preferential Subspace Identification (PSID) model \parencite{saniModelingBehaviorallyRelevant2021} and the nonlinear Dissociative Prioritized Analysis of Dynamics (DPAD) model \parencite{saniDissociativePrioritizedModeling2024}. While both have shown success in capturing behaviorally relevant neural dynamics in animal studies, their application to pathological human signals remains unexplored.

The project will benchmark these models across three core research questions: (1) cross-modal prediction of cortical ECoG from subcortical LFP signals, (2) classification of discrete DBS ON/OFF brain states, and (3) continuous decoding of motor behavior (tracing speed). Baseline configurations, including Neural Dynamic Modeling (NDM) and a linear DPAD variant, will be implemented to enable comparison of analytical versus numerical optimization, standard versus specialized modeling, and linear versus nonlinear representations.

By applying these frameworks to simultaneous LFP and ECoG data from PD patients, this research aims to extract behaviorally relevant dynamics from multiplexed neural signals. The findings may provide new insights into the neural mechanisms of PD and inform the development of robust biomarkers for future closed-loop stimulation therapies.

\subsection{Data and Preprocessing}
The dataset for this project is from the Dareplane project \parencite{doldLFPECoGData2024}, consisting of simultaneous 16-channel Local Field Potential (LFP) data from bilateral subthalamic nucleus (STN) electrodes and 4-channel Electrocorticography (ECoG) data from the primary motor cortex, recorded from 4 participants with Parkinson's disease across 9 sessions. Additionally, there are 20 sessions of EEG data from 8 participants. During the LFP/ECoG acquisition, hand kinematics were also captured, and the tracing speed will be calculated from coordinate changes through time. LFP and ECoG data were originally sampled at 22 kHz and contain separate blocks corresponding to DBS ON and OFF states.

A specific preprocessing pipeline will be applied to each signal modality after downsampling LFP and ECoG data to 1000 Hz and EEG to 100Hz and applying a notch filter to remove power-line noise (50 Hz and its harmonics).

\begin{itemize}
    \item \textbf{LFP Processing:} Initial preprocessing will focus on removing high-amplitude electrical artifacts from the DBS-ON recordings using a template subtraction method \parencite{qianMethodRemovalDeep2017,hammerArtifactCharacterizationMultipurpose2022}. The cleaned LFP data will then be band-pass filtered (3-250 Hz) and re-referenced using a Common Average Reference (CAR).

    \item \textbf{ECoG Processing:} The ECoG data will be band-pass filtered (3-250 Hz, including high gamma) and re-referenced using a Common Average Reference (CAR).

    \item \textbf{EEG Processing:} The EEG data will be preprocessed following the pipeline detailed in the relevant literature for this dataset. This will include band-pass filtering, re-referencing, and the use of Independent Component Analysis (ICA) to identify and remove physiological artifacts. To derive task-specific spatial filters that isolate brain activity related to motor behavior, Source Power Comodulation (SPoC) will be employed.
\end{itemize}

For the primary analysis, the PSID and DPAD models will be trained directly on segmented one-second epochs of the preprocessed LFP and ECoG time-series data.

To assess model generalization, a leave-one-session-out cross-validation scheme, stratified by participant, will be employed. For each fold, a model will be trained and validated on data from all but one session of a given participant, with the held-out session serving as the final test set. This process will be repeated for all sessions across all participants to ensure that the model is always tested on data it has not seen from a specific day. Within the training portion of each fold, the data will be further subdivided into a training set (80\%) and a validation set (20\%).


\subsection{Exploratory Data Analysis}
Before addressing the primary research questions, a series of exploratory data analyses will be conducted to characterize the statistical properties of the dataset. These initial analyses will serve to validate the premises of the subsequent modeling work. First, the relationship between the recorded neural signals (LFP, ECoG) and the behavioral variable (tracing speed) will be quantified using both Pearson and Spearman correlation coefficients to assess linear and monotonic trends, respectively. Second, to establish a baseline for the cross-prediction task in RQ1, the relationship between the neural modalities will be analyzed by calculating the correlation between LFP and ECoG signals. Third, Power Spectral Density (PSD) estimates will be computed for different conditions (DBS ON vs. OFF) and statistically compared to provide initial evidence for the brain state classification task in RQ2. Finally, after model fitting, diagnostic checks will be performed on the model residuals. This analysis will test for properties such as whiteness (i.e., lack of autocorrelation) to validate that the models have adequately captured the underlying neural dynamics.

\subsection{Preferential Subspace Identification (PSID)}
To address it, \textcite{saniModelingBehaviorallyRelevant2021} developed Preferential Subspace Identification (PSID), a method that directly targets these shared dynamics. The main principle of PSID is that by training a model to predict future behavioral outputs from past neural activity, the model is forced to discover and represent the behaviorally relevant neural dynamics. To formalize this, PSID adopts a linear time-invariant (LTI) state-space model structure, which provides a tractable and interpretable foundation:
\[
    \begin{cases}
        \mathbf{x}_{k+1} = A \mathbf{x}_k + \mathbf{w}_k \\
        \mathbf{y}_k = C_y \mathbf{x}_k + \mathbf{v}_k   \\
        \mathbf{z}_k = C_z \mathbf{x}_k + \boldsymbol{\epsilon}_k
    \end{cases}
\]
Here, $k$ is the time index; $\mathbf{x}_k \in \mathbb{R}^{n_x}$ is the unobserved, low-dimensional latent state that evolves according to the state transition matrix $A$; $\mathbf{y}_k \in \mathbb{R}^{n_y}$ is the observed neural activity from $n_y$ channels, generated via the observation matrix $C_y$; and $\mathbf{z}_k \in \mathbb{R}^{n_z}$ is the observed behavioral variable, generated via the observation matrix $C_z$. The terms $\mathbf{w}_k$, $\mathbf{v}_k$, and $\boldsymbol{\epsilon}_k$ represent state, neural observation, and behavioral residuals, respectively.

The PSID algorithm learns the model parameters and latent states through a non-iterative, closed-form procedure. It begins by constructing matrices of past neural activity ($\mathbf{Y}_p$) and future behavior ($\mathbf{Z}_f$). The "preferential" step is computing the orthogonal projection of future behavior onto past neural activity,
\[
    \hat{\mathbf{Z}}_f = \mathbf{Z}_f \mathbf{Y}_p^T (\mathbf{Y}_p \mathbf{Y}_p^T)^{-1} \mathbf{Y}_p
\]
which isolates the component of future behavior that is linearly predictable from the neural signals. This projection is then decomposed using Singular Value Decomposition (SVD) to identify the observability matrix and the behaviorally relevant latent states. From this optimally identified subspace, the system matrices are estimated via linear regression. Specifically, the state transition matrix ($A$) is found by regressing future latent states onto current latent states, while the observation matrices ($C_y$, $C_z$) are found by regressing the observed neural and behavioral data onto the estimated latent state sequence. Finally, a Kalman filter is applied with these estimated parameters to compute the optimal sequence of the latent state $\mathbf{x}_k$. The final output is a low-dimensional, interpretable linear model that captures the neural dynamics most relevant to the behavior of interest.

\subsection{Dissociative Prioritized Analysis of Dynamics (DPAD)}

While PSID provides a robust linear framework, it is well-established that the brain's computations are fundamentally nonlinear. \textcite{saniDissociativePrioritizedModeling2024} proposed a new neural dynamics modelling framework called Dissociative Prioritized Analysis of Dynamics (DPAD). Although it may be reminiscent to PSID, but it is mathematically distinct from PSID. DPAD not only \textit{prioritizes} behaviorally relevant dynamics but also explicitly \textit{dissociates} them from irrelevant dynamics. This is achieved by partitioning the model's latent state $\mathbf{x}_k$ into two separate components: a prioritized state, $\mathbf{x}_k^{(1)}$, which is dedicated to predicting behavior, and a non-prioritized state, $\mathbf{x}_k^{(2)}$, which models all remaining neural variance.
This dissociation is enforced through a four-step training process. The model is structured such that the behavior, $\hat{\mathbf{z}}_k$, is predicted only from the prioritized latent state, whereas the neural activity, $\hat{\mathbf{y}}_k$, is reconstructed from both states.
$$
    \begin{cases}
        \mathbf{x}_{k+1} = \mathbf{A}'(\mathbf{x}_k) + \mathbf{K}'(\mathbf{y}_k)           \\
        \hat{\mathbf{y}}_k = C_y^{(1)}(\mathbf{x}_k^{(1)}) + C_y^{(2)}(\mathbf{x}_k^{(2)}) \\
        \hat{\mathbf{z}}_k = C_z^{(1)}(\mathbf{x}_k^{(1)})
    \end{cases}
$$
Here, $\mathbf{A}'$, $\mathbf{K}'$, $C_y^{(1)}$, $C_y^{(2)}$, and $C_z^{(1)}$ are now nonlinear functions (e.g. MLPs) that represent the state recursion, neural input, and observation mappings, respectively

First, the entire network is trained by forcing the prioritized section of the RNN and its latent state $\mathbf{x}_k^{(1)}$ to learn only the dynamics predictive of behavior. Second, the weights of this "prioritized" section are frozen. Third, the optimization objective shifts entirely to neural reconstruction, compelling the second, non-prioritized section of the RNN ($\mathbf{x}_k^{(2)}$) to learn and explain the residual neural variance not captured by the first stage. Finally, an optional fine-tuning step adjusts all parameters jointly. This process yields a nonlinear model that not only achieves high predictive accuracy but also provides an interpretable, dissociated latent representation of the neural dynamics, making it a suitable framework for exploring the complex dynamics of Parkinson's disease.


\subsection{Baseline Model Configurations}
To benchmark the performance of the primary models, two key baselines will be implemented to represent standard modeling approaches and to isolate the effects of numerical optimization.

\subsubsection{Neural Dynamic Modeling (NDM) Baseline}

The NDM baseline, which models neural dynamics agnostic to behavior, will be configured for both frameworks. This is achieved by setting the dimension of the behaviorally-prioritized subspace to zero ($n_1 = 0$).
    \begin{itemize}
        \item For \textbf{PSID}, setting $n_1=0$ means that no latent states are extracted by projecting future behavior; instead, the states are identified by projecting future \textit{neural activity} onto past neural activity, consistent with standard Subspace Identification (SID).
        \item For \textbf{DPAD}, setting $n_1=0$ bypasses the first two optimization steps that rely on behavioral data. The model's entire objective then becomes learning a latent state that best predicts future neural activity, which is the definition of an NDM.
    \end{itemize}

\subsubsection{Linear DPAD Baseline}
To create a linear model that uses numerical optimization (serving as a direct counterpart to the analytical PSID), a linear version of DPAD will be configured. This will be achieved by setting the neural input ($K'$), state recursion ($A'$), and observation mappings ($C_y, C_z$) to be linear. Within the underlying recurrent neural network architecture, this will be implemented by specifying zero hidden layers for each of these components in the model's nonlinearity settings. This reduces their operations to linear matrix transformations, making the overall model a linear state-space system that is trained via numerical optimization.

\subsection{RQ1: Cross-Modal Neural Prediction}
\begin{quote}
    \textbf{Research Question 1:} To what extent can the models predict future ECoG activity from past ECoG activity? Building on this, to what extent can the models predict cortical ECoG activity from subcortical LFP recordings, and how well do these predictions generalize across recording sessions for a given patient?
\end{quote}

This question will investigate the feasibility of neural signal translation. The initial configuration consists of self-prediction, whether it is possible to predict ECoG signals a head of time. The models will be trained to predict ECoG using only LFP input, focusing on data from the DBS-OFF state. Prediction accuracy will be quantified using the Coefficient of Determination ($R^2$). For a deeper analysis of the learned relationship, Canonical Correlation Analysis (CCA) and Deep CCA (DCCA) will be employed to quantify the strength of the linear and nonlinear coupling, respectively, between the LFP and ECoG signals, and between past ECoG and predicted ECoG.

\subsection{RQ2: Brain State Classification}
\begin{quote}
    \textbf{Research Question 2:} How accurately can PSID and DPAD classify discrete brain states (DBS ON vs. OFF) from LFP/ECoG signals?
\end{quote}

This task assesses the models' ability to identify distinct, clinically relevant brain states. The methodology involves using the learned latent states as features for a logistic regression classifier to perform a binary classification of the DBS ON/OFF condition. Performance will be assessed using metrics, including the Area Under the ROC Curve (AUC), the F1-Score, and Balanced Accuracy. To interpret the learned representations, the latent state trajectories will be visualized using t-SNE to confirm class separability. To quantitatively compare the learned representations, Representational Similarity Analysis (RSA) will be performed. Representational Dissimilarity Matrices (RDMs) will be constructed from both the neural data and the models' latent states, and their correlation will be calculated to assess the alignment between the model's internal geometry and the underlying neural geometry.
\subsection{RQ3: Continuous Motor Behavior Decoding}
\begin{quote}
    \textbf{Research Question 3:} How effectively can PSID and DPAD decode continuous motor behavior (tracing speed) from LFP activity, and what do their respective latent dynamics reveal about the linear vs. nonlinear neural control of movement?
\end{quote}

This final question addresses the core challenge for developing future adaptive DBS systems. The models will be trained to decode a continuous behavioral variable, tracing speed, from LFP signals. Decoding performance will be evaluated using Pearson's correlation coefficient. The quantitative link between the models' internal dynamics and the behavior will be established using CCA and DCCA to measure the correlation between the latent state trajectories and the continuous tracing speed data. For the DPAD model, the distinct contributions of its prioritized ($\mathbf{x}^{(1)}$) and non-prioritized ($\mathbf{x}^{(2)}$) subspaces will be explicitly compared. Furthermore, RSA will be used to compare the geometric structure of the latent spaces to the structure of the behavior itself.


\vspace{1em}
\noindent\textbf{Word Count:} 2551
\newpage
\section{SCHEDULE}
\begin{table}[h!]
    \centering
    \label{tab:schedule}
    \resizebox{0.95\textwidth}{!}{
        \begin{tabularx}{\textwidth}{>{\RaggedRight}X l l >{\RaggedRight}X}
            \textbf{Task}                                        & \textbf{Start Date} & \textbf{End Date} & \textbf{Key Goals / Milestones}                                                                                             \\
            \midrule
            Finalizing project proposal                          & 2025-09-01          & 2025-09-07        &                                                                                                                             \\

            Data Analysis \& pipeline scaffolding                & 2025-09-08          & 2025-09-28        & Pipeline for DBS/ECoG preprocessing                                                                                         \\

            Building full training pipeline                      & 2025-09-29          & 2025-11-02        & Pipeline running a full experiment (load, preprocess, train, evaluate) from a YAML configs.                                 \\

            Drafting Background \& Methodology                   & 2025-09-29          & 2025-11-02        &                                                                                                                             \\

            Baselines model training                             & 2025-11-03          & 2025-11-16        & Generate first-pass results with PSID and a basic DPAD configuration on all RQs.                                            \\

            Main Training \& Hyperparameter Tuning               & 2025-11-10          & 2025-12-21        & Running all the experiments with different configurations.                                                                  \\

            In-depth Analysis of Initial Results                 & 2025-11-17          & 2025-12-07        & Analyze baseline results. Develop scripts for latent space analysis, create initial figures.                                \\


            Refining Background \& Methods Drafts                & 2025-12-08          & 2025-12-21        & Incorporate new insights from initial results into the report drafts.                                                       \\

            Christmas Break                                      & 2025-12-22          & 2026-01-04        & Rest and light work on refining drafts as needed.                                                                           \\

            Getting full results                                 & 2026-01-05          & 2026-01-18        & Finalize all model training and hyperparameter tuning. Get all the results and ensure they are reproducible and documented. \\

            Drafting Results Section                             & 2026-01-05          & 2026-01-25        & Write the main narrative for the Results section using all finalized model outputs.                                         \\

            Drafting Discussion Section                          & 2026-01-19          & 2026-02-08        & Connect results back to the background and discuss implications.                                                            \\

            Planning Thesis Presentation                         & 2026-01-26          & 2026-02-08        &                                                                                                                             \\

            Drafting Introduction \& Abstract                    & 2026-02-09          & 2026-02-15        &                                                                                                                             \\

            Consolidating Full Report                            & 2026-02-15          & 2026-02-22        & Polish all sections, figures, and references into a single, cohesive document.                                              \\

            Finalizing Thesis Presentation/Slides                & 2026-02-23          & 2026-03-01        & Finalize slides and practice the presentation.                                                                              \\

            Final Revisions, Unplanned Delays \& Submission Prep & 2026-03-02          & 2026-03-30        &                                                                                                                             \\
        \end{tabularx}
    }
\end{table}

\newpage

\section{SCIENTIFIC, SOCIETAL AND/OR TECHNOLOGICAL RELEVANCE}
% TODO: Describe the broader context and relevance of your project.
% : (ABOUT 250 WORDS)
The clinical implications of this research extend directly to adaptive Deep Brain Stimulation (aDBS) for Parkinson's disease, aiming to transform the current standard of continuous stimulation into an intelligent, on-demand therapy. By validating computational models capable of decoding motor performance and classifying clinical states in real-time, .this work aims to provide an analytical foundation for future closed-loop systems that could significantly enhance therapeutic efficacy while mitigating side effects. Furthermore, the exploration of cross-modal prediction investigates the potential for creating "virtual sensors," which could leverage the rich information from cortical signals without the need for additional invasive surgeries. However, the broader implications of this methodological framework are not confined to Parkinson's disease or motor control. The ability to dissociate and model behaviorally-relevant neural dynamics is a fundamental tool applicable to a wide range of neurological and psychiatric conditions, including epilepsy, depression, and obsessive-compulsive disorder. The same principles could be used to decode seizure precursors, affective states, or cognitive fluctuations, paving the way for adaptive neuromodulation therapies across the clinical spectrum.

\section{REFERENCES}
\printbibliography[heading=none]


\end{document}